\documentclass[UTF8]{ctexart}
\usepackage[T1]{fontenc}
\usepackage[utf8]{inputenc}
\usepackage{authblk}
\usepackage{graphicx}
\usepackage{booktabs}
\usepackage{makecell}
\usepackage{subfigure}
\usepackage{float}
\usepackage{amsmath}
\usepackage{listings}
\usepackage{xcolor}
\lstset{
    numbers=left, 
    numberstyle= \tiny, 
    keywordstyle= \color{ blue!70},
    basicstyle=\footnotesize,
    commentstyle= \color{red!50!green!50!blue!50}, 
    frame=shadowbox, % 阴影效果
    rulesepcolor= \color{ red!20!green!20!blue!20} ,
    escapeinside=``, % 英文分号中可写入中文
    %basicstyle=\script,
    xleftmargin=2em,xrightmargin=0em, aboveskip=1em,
    framexleftmargin=2em
} 


\begin{document}

\begin{titlepage}
        \heiti
        \vspace*{64pt}
        \begin{center}
           
            \fontsize{24pt}{24pt} {天气应用设计与建模实验报告} \\
	\vspace*{24pt}   
            \vspace*{48pt}
            \vspace*{48pt}
        
            \vspace*{72pt}
        
	   \Large  姓名:\ \ \underline{\makebox[216pt]{孙一鸣 }}

	   \Large  学号:\ \ \underline{\makebox[216pt]{221220033}}

	\vspace*{48pt}
	 \Large 日期:\ \ \underline{\makebox[216pt]{2024年10月11日}}

        \end{center}
    \end{titlepage}
\newpage

\section{软件整体架构}

根据该软件的功能需求,将软件大致划分为以下模块:

\begin{center}
\begin{tabular}{| c| c|  }
\hline
模块 & 功能  \\ 
\hline\hline
 Weather Module & 处理当前天气、湿度、多城市天气查询等 \\  
\hline 
Forecast Module & 处理2小时内的天气预测功能 \\
\hline
Air Quality Module & 处理空气质量和健康建议功能  \\
\hline
Alert Module & 处理实时天气警报和极端天气通知 \\
\hline
Customization Module & 处理用户的主题、布局自定义设置及个性化通知\\
\hline
Travel Mode Module &  提供旅行路径天气预测和建议\\
\hline
Widget Module & 实现主屏幕的小部件功能\\
\hline
User Module & 管理用户的个人信息及偏好设置\\
\hline
Notification Module & 处理短信警告和其他通知机制 \\
\hline
\end{tabular}
\end{center}

根据这一架构,绘制该软件的类图:

 \subsection{ 类图}

\begin{figure}[htbp]
\centering
\includegraphics[scale=0.2]{lei}
\caption{类图}
\label{figure}
\end{figure}

在这一类图中,weather是核心模块,负责与forecast, airquality, alert, travelmode模块交互,并调用这些模块来实现对应功能;forecast模块负责2小时内的天气预测;airquality模块负责处理空气质量和健康建议;alert模块负责极端天气预警;travelmode模块负责旅行模式,提供沿途天气和出行建议。

在此之外,customization模块负责个性化设置,包括主题、布局、通知设置等,并与user模块交互;user模块,alert模块共同与notification模块交互,从而使用户可以接受到极端天气警告。widget模块与customzation模块交互,允许用户自定义桌面天气组件。

\subsection{组件图}

\begin{figure}[htbp]
\centering
\includegraphics[scale=0.2]{zu}
\caption{组件图}
\label{figure}
\end{figure}

在该图中,WeatherService是核心组件,负责天气相关信息的数据获取和处理;UserService负责与用户相关的个性化服务和小工具设置;AlertService处理极端天气警报,并通过NotificationService象用户发送通知;各组件均与用户界面(UI)紧密交互,以提供实时更新和数据展示。

\subsection{活动图}

\begin{figure}[htbp]
\centering
\includegraphics[scale=0.2]{huo}
\caption{活动图}
\label{figure}
\end{figure}

在这一活动图中,用户打开APP开始使用后,选择城市或者使用定位功能获取位置信息;之后APP向后台服务请求天气数据;系统检查用户是否登录,如果已登录,加载用户的个性化设置;否则使用默认设置。APP调用 WeatherService,从不同的模块获取天气、湿度、空气质量、未来天气预测等信息;并且检查是否有极端天气,如果有则触发 AlertModule,发送通知警报;如果没有,则跳过此步骤。系统将各个模块获取的数据整合,并应用用户的主题和布局设置;最后,将整合后的天气信息呈现给用户,结束流程。

\subsection{通信图}

\begin{figure}[htbp]
\centering
\includegraphics[scale=0.2]{tong}
\caption{通信图}
\label{figure}
\end{figure}

该通信图展示了各个模块在用户查询天气时的交互过程,用户通过 UI 触发各服务模块,获取天气、预测、空气质量及警报信息。CustomizationModule 和 NotificationModule 用于个性化显示和通知功能。
\newpage

\section{UI界面}

\begin{figure}[htbp]
\subfigure[UI1]{
\begin{minipage}{6cm}
\centering
\includegraphics[scale=0.5]{UI1}
\end{minipage}
}
\subfigure[UI2]{
\centering
\begin{minipage}{6cm}
\centering
\includegraphics[scale=0.5]{UI2}
\end{minipage}
}
\end{figure}

UI1是该天气APP的主界面,在这个界面上实现了Weather Module和Forecast Module 的内容,可以看到当前的天气情况,温度,湿度,以及未来数小时内的天气、温度状况,并使用折线图给出直观的展示。另外在这一界面还可以看到当前城市信息。


点击主界面的城市按钮,可以来到城市选择(UI2)界面,可以在搜索框直接输入城市名,或者在下方通过城市名首字母检索。

\newpage
\begin{figure}[htbp]
\subfigure[UI3]{
\begin{minipage}{6cm}
\centering
\includegraphics[scale=0.5]{UI3}
\end{minipage}
}
\subfigure[UI4]{
\centering
\begin{minipage}{6cm}
\centering
\includegraphics[scale=0.5]{UI4}
\end{minipage}
}
\end{figure}

UI3界面是空气质量界面,实现了Airquality Module,可以查看当前空气质量整体状况和健康建议;并附有污染物的详细信息。

极端天气预警界面实现了Alert Module,可以显示即将到来的极端天气及到来时间。
\newpage
\begin{figure}[htbp]
\subfigure[UI5]{
\begin{minipage}{6cm}
\centering
\includegraphics[scale=0.5]{UI5}
\end{minipage}
}
\subfigure[UI6]{
\centering
\begin{minipage}{6cm}
\centering
\includegraphics[scale=0.5]{UI6}
\end{minipage}
}
\end{figure}

用户界面实现了UserModule和 Customzation Module,用户可以在这一界面自定义APP的颜色,布局,以及背景图片;还可以管理通知方式。

UI6界面是Travelmode Module界面,用户可以输入旅行起点与终点,APP会自动设计最佳路径,并显示沿途城市的天气状况,提供出行建议。
\newpage
\begin{figure}[htbp]
\centering
\includegraphics[scale=0.5]{UI7}
\caption{UI7}
\label{figure}
\end{figure}

UI7界面是用户手机桌面的小组件,实现Widget Module,用户可以便捷地在桌面直接看到天气的大致信息。

\section{大模型使用情况}

在绘制UML图时,我让Chatgpt分别给出了类图、组件图、活动图、通信图的示例代码,并在其基础上实现了基于实验一需求的UML图绘制。
\begin{figure}[htbp]
\centering
\includegraphics[scale=0.5]{chat}
\caption{Chat with chatgpt}
\label{figure}
\end{figure}

我根据gpt给出的示例代码了解了这4种UML图的绘制方式,结合需求后自行绘制了前文提过的UML图。



\end{document}